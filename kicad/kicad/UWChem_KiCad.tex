\documentclass[12pt]{article}

% geometry
\usepackage[margin=0.75in]{geometry}

% text
\usepackage{setspace}
\renewcommand{\familydefault}{\sfdefault}
\setlength\parindent{0pt}
\setlength{\parskip}{\baselineskip}

% lists
\usepackage[shortlabels]{enumitem}
\setlist[enumerate, 1]{nosep}
\setlist[enumerate, 2]{nosep, topsep=-5ex}
\setlist[enumerate, 3]{nosep, topsep=-5ex}
\setlist[enumerate, 4]{nosep, topsep=-5ex}
\setlist[itemize, 1]{nosep}
\setlist[itemize, 2]{nosep, topsep=-5ex}
\setlist[itemize, 3]{nosep, topsep=-5ex}
\setlist[itemize, 4]{nosep, topsep=-5ex}
\newenvironment{ditemize}
  {
  \begin{itemize}
  \renewcommand{\labelitemi}{$\rightarrow$}
  }
  {
  \end{itemize}
  }

\newcommand{\ra}{$\rightarrow$}

\usepackage{graphicx}

\usepackage[yyyymmdd]{datetime}
\renewcommand{\dateseparator}{--}

\begin{document}

% title
\title{UW-Madison Chemistry \\ Introduction to KiCad}
\date{\today}
\author{Blaise Thompson}
\maketitle

\section{electronic computer-aided design}

Various options, including:
\begin{ditemize}
  \item EAGLE (part of autodesk family)
  \item ExpressPCB
  \item KiCAD
  \item Altium
  \item Cadence OrCAD
\end{ditemize}
Prefer KiCad.

Most ECAD programs have two main interfaces to the circuit:
\begin{enumerate}
  \item The schematic
  \item The PCB
\end{enumerate}

\clearpage
\section{introduction to KiCad}

download from http://kicad-pcb.org/download/

When you first launch KiCad, you will be met with a totally blank application.

KiCad is actually made up of several sub-programs.
You will see these listed along the top of the page.
They include:
\begin{ditemize}
  \item schematic layout editor
  \item symbol library editor
  \item PCB layout editor
  \item footprint library editor
  \item gerber viewer
  \item import bitmap
  \item calculator
  \item worksheet layout editor
\end{ditemize}

For this introduction, we will be building a simple delay generator circuit from 628.

\subsection{creating a project}

Before we create our first project, a word about staying organized.
A KiCad project is not a single file.
Instead, a KiCad project is a \emph{folder} with different files corresponding to different pieces of your project, like the schematic and (when appropriate) PCB.
Furthermore, you will find yourself using and modifying \emph{libraries} of electronic footprints and symbols---these will also become part of your project.
It is imperative that the contents of this folder remain organized, or you risk ``breaking'' your project.

Use \textbf{File \ra{} New \ra{} Project} to create a new project.
Choose the location and name of your new project.
You will find that KiCad creates a folder at your chosen location.
That folder is immediately populated with three files:
\begin{ditemize}
  \item kicad\_pcb
  \item pro
  \item sch
\end{ditemize}


\clearpage
\section{schematic capture using eeschema}


Change page layout.

\clearpage
\section{pcb layout using pcbnew}

Change page layout.

\end{document}
