\documentclass{training}

\title{PCB etching}
\date{\today}
\author{Helena Pliszka, Blaise Thompson}

\begin{document}

\maketitle
\renewcommand{\baselinestretch}{0.5}\normalsize
\tableofcontents
\renewcommand{\baselinestretch}{1.0}\normalsize
\vfill

Part of the training materials prepared by the \href{https://shops.chem.wisc.edu/}{Chemistry Shops} at UW--Madison. \\
This document was prepared using \href{https://www.latex-project.org/}{\LaTeX}. \\
Source code and all associated files can be found at \href{https://git.chem.wisc.edu/shop/training/pcb-etching}{git.chem/shop/training/pcb-etching}. \\
If you find any mistakes or feel that any information is missing, please \href{https://git.chem.wisc.edu/shop/training/pcb-etching/issues}{open an issue}. \\

\clearpage
\section{design considerations}

We have the ability to make printed circuit boards (PCBs) in house.
The end product is not typically appropriate for long term use, but the very short turn-around time makes in-house printing an invaluable prototyping tool.

We can print circuit boards up to 150 x 230 mm (6 x 9''). Printing cost depends on the size of the board: \\
- quarter (up to 75 x 115 mm) \$20 \\
- half (up to 115 x 150 mm) \$30 \\
- full (up to 150 x 230 mm) \$40 \\
These flat prices include the cost of materials and staff labor. \\
Typically, boards can be printed in one day or less.

Use a PCB layout program like KiCAD to plan your board---ask a staff member for help if you are unfamiliar with these programs.
We use photolithography to create our boards.
Our boards are limited to two layers, and do not have vias.
With these limitations in mind, the following guidelines are recommended when designing your prototype: \\
- use 35 mil (or larger) traces for power and common \\
- use 15 mil (or larger) traces for signal \\
- for typical components, use pads with 50 mil outer diameter and 39 mil hole diameter \\
- many components cannot be soldered from the top of the PCB---for these, ensure that connecting traces \\ \phantom{-} approach from the bottom \\
- avoid right angles in your traces, use two 45 degree bends instead \\
- mark 0.125'' diameter circles at board corners for \#4 standoff mounting screws \\
Once you are finished laying out your board, email it to Blaise (bthompson@chem.wisc.edu) for printing.

You will receive your board without any holes.
Use the a drill press to make holes for all of your components.
Ask a staff member for help if you are unfamiliar with this process.
Drill all your holes and dry-fit your components before doing any soldering.

Typical components require a \#65 drill bit.
For 4-40 standoff mounting screws, use the large drill press and a 1/8'' bit.
When in doubt, use a pair of calipers to measure the component legs and use the next largest drill in the following table:

\begin{center}
\begin{tabular}{c | c | l}
  drill bit & diameter & typical usage \\ \hline
  70 & 28 mil & integrated circuits, sockets \\
  65 & 35 mil & discrete resistors, capacitors, diodes etc \\
  60 & 40 mil & wire, large components \\
  1/8 & 0.125'' & \#4 machine screw through holes
\end{tabular}
\end{center}

Enjoy your prototype!
We hope that it works exactly as expected.

\clearpage
\section{standard operating procedure}

Personal protective equipment:
\begin{itemize}
  \item lab coat
  \item goggles
  \item gloves
\end{itemize}

Setup:
\begin{itemize}
  \item Turn on heating for copper etchant bath, since takes a while to warm up
Align transparencies on well lit fume hood window and tape together with scrap board in between to mimic board thickness
  \item Turn off light sooner to let your eyes adjust (once don’t have to leave room anymore)
Pour about ½ inch of 9:1 developer into bin
\end{itemize}

Board Exposure (room must be dark for all steps)
\begin{itemize}
  \item Look through previously cut pieces for size match -- if need be trim with heavy duty paper cutter
  \item Peel off protective sticker (will take off some photoresist too unfortunately)
  \item Place in between transparencies and lay in UW exposure box (carefully open/close box, lid is very heavy)
  \item Leave to expose for 6 minutes on dial
\end{itemize}

Developing
\begin{itemize}
  \item Turn room light on
  \item Remove board from transparency enclosure and set in developing solution
  \item With your thumbs holding the board in place, slosh the solution over the board (if the board slides, it will be scratched)
  \item Flip board to ensure even washing
  \item Watch blue photoresist wash off
\end{itemize}

Quenching
\begin{itemize}
  \item Rinse board gently with DI water (comes from hose)
\end{itemize}

Etching:
\begin{itemize}
  \item Loosen one screw on etchent bath holder
  \item Tighten the circuit between the two plastic channels, resting it on the lower metal stops
  \item Tighten screw, ensuring circuit will not fall out of place
  \item With Na2S2O8 bath around 40C, lower circuit into solution and switch on “stirrer” that will bubble air through the bath
  \item Check every 15 minutes, for roughly an hour till copper color goes from gold → salmon pink → yellow (make sure not to overexpose)
\end{itemize}

Finalization
\begin{itemize}
  \item dry circuit
  \item cut to size
  \item deliver to customer
\end{itemize}

\end{document}
