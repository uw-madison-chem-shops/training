\documentclass{training}
\usepackage{multicol}

\title{Instrument Shop \\ Repair Policy}
\date{\today}
\author{Blaise Thompson, Matt Martin, Steve Myers, Amber Bartz}

\begin{document}

\maketitle
\renewcommand{\baselinestretch}{0.5}\normalsize
\tableofcontents
\renewcommand{\baselinestretch}{1.0}\normalsize
\vfill

Part of the training materials prepared by the \href{https://shops.chem.wisc.edu/}{Chemistry Shops} at UW--Madison. \\
Source code and all associated files can be found at \href{https://github.com/uw-madison-chem-shops/training}{GitHub}. \\
If you find any mistakes or feel that any information is missing, please \href{https://github.com/uw-madison-chem-shops/training/issues}{open an issue}. \\

\clearpage

\section{Repair Policy}

The Instrument Shop commonly repairs laboratory instruments, appliances, and accessories.
In most cases, repair is a safe and effective way to increase the usable lifetime of commercial scientific equipment and ensure continuity of operations.
However, there are cases where equipment cannot be repaired.

When repair is impossible, shop staff will attempt to recommend replacement options, including the possibility of constructing custom replacements.
When critical research is impeded, the Shop will prioritize working with researchers to find a replacement quickly.

\subsection{Chemical Contamination}

For the safety of Instrument Shop staff, instruments and appliances submitted for repair must be free of chemical contamination.
Any instrument/appliance that has been exposed to research chemicals must be cleaned by the researcher.
Use the standard UW-Madison “OK to Repair” procedure and associated sticker/tag to make it clear that the instrument/appliance has been cleaned.
Broken instruments or appliances that cannot be sufficiently cleaned must be disposed of through chemical safety.

As a special case, vacuum pumps do not use the “OK to Repair” labels.
Instead, customers must fill out and sign the vacuum pump repair declaration statement at time of submission.
In certain cases, customers may be asked to drain pump oil before submission to the Shop.

\subsection{Dangerous Instruments}

The Instrument Shop reserves the right to refuse repair of instruments and appliances which, in our opinion, do not meet basic electrical safety standards.
In some cases, the instrument/appliance was originally constructed in a way that does not meet modern standards.
In other cases, instruments/appliances may be damaged to such an extent that they become unsafe for continued use.

The Instrument Shop will absolutely refuse to work on instruments/appliances with improper electrical shielding.
Instruments and appliances typically must have over-current protection such as fuses or internal breakers.
In certain cases, the Instrument Shop will strongly suggest incorporation of GFCI or over-temperature
protection mechanisms.
In many cases, modern replacements are available which incorporate these safety features.

In cases where an instrument/appliance is clearly too dangerous for continued use, the Shop may refuse to return the instrument/appliance to the customer.
Any disagreements will be resolved by the departmental safety committee.

\subsection{Liability and Insurance Markings}

Commercially sold instruments/appliances are typically certified for insurance purposes (UL or otherwise).
Repair work by the Instrument Shop voids this certification.
For audit and oversight purposes, the Instrument Shop will permanently remove or deface the certification sticker/mark and install a label indicating that the device has been repaired.
The Shop will attempt to keep records of each repair job for auditing purposes.
You may request these records at any time.

Because of the liability involved in repair, the Instrument Shop will in-general refuse electrical repair for originally-certified instruments/appliances that cost less than \$100 to replace.

\subsection{Challenging and Complex Repair}

Certain repair jobs are much harder than others.
Instrument Shop staff will do their best to accomplish all repairs, but certain repair jobs will be beyond the capabilities of our small shop.

In general, the Instrument Shop does not attempt circuit-board-level repair.
This type of repair requires full reverse-engineering of the instrument/appliance, a process that can take tens to hundreds of hours.
The Shop may request that customers contact original manufacturers to obtain schematics before complex repair is attempted.
Shop staff will assist customers in finding schematics, when appropriate.

Often, repair of specialty instrumentation requires sourcing unusual components for replacement. Sometimes, finding these components is a labor-intensive process that goes beyond Shop capabilities.
In these cases, the Shop will ask customers to obtain specialty parts before attempting repair.

\subsection{Special Cases}

The Instrument Shop reserves the right to refuse repair for any reason not listed here.
The Shop will always make it clear why repair has been refused.
Any disagreements will be resolved by the departmental shops committee.

\clearpage
\section{Repair Submission Checklist}

\subsection{Small Appliances}

The Instrument Shop regularly repairs small appliances used in chemical synthesis, including:

\begin{multicols}{3}
\begin{itemize}
  \item hotplates
  \item stirplates
  \item shakers
  \item laboratory ovens
  \item rotovaps
  \item UV lamps
  \item sonicators
  \item balances
  \item chillers
\end{itemize}
\end{multicols}

To submit a small appliance for repair:

\begin{enumerate}
  \item Pick up an ``OK to move / Repair'' label from the sub-basement hallway.
  \item Clean the appliance using appropriate methods, as described by the ``OK to move / Repair'' poster.
  \item Attach the ``OK to move / Repair'' label to your appliance.
  \item Transport the item from your laboratory to the sub-basement, leaving it in the hallway. Place it on the pink shelves if it fits, otherwise you may leave it on the floor or on a cart. If the appliance is too large to be moved, contact Shop staff for assistance.
  \item Send an email to Amber Bartz  (afbartz@wisc.edu), letting her know that the appliance is in the hallway in need of repair. Share any details about how the appliance has malfunctioned.
  \item Shop staff will contact you.
\end{enumerate}

\subsection{Vacuum Pumps}

The Instrument Shop has special capabilities dedicated to vacuum pump repair.
To submit a pump for repair:

\begin{enumerate}
  \item Transport the pump from your laboratory to the sub-basement, leaving it in the hallway. If the pump is too large to be moved, contact Shop staff for assistance.
  \item Fill out the vacuum pump repair job card, which you can find in the sub-basement.
  \item Send an email to Matt Martin (mdmartin2@wisc.edu), letting him know that the pump is in the hallway in need of repair. Share any details about how the pump has malfunctioned.
  \item Shop staff will contact you.
\end{enumerate}

\subsection{Specialty Devices}

The Instrument Shop often attempts repair for specialty devices, including specialty instrumentation and unique appliances.
If you have a unique device needing repair, please contact Shop staff via email.
We will work with you to figure out what's possible.

\end{document}
